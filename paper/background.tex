
\section{Background}
Modern CPU designs employ a wide range of tricks in order to maximize
performance. In this section, we provide preliminary background as they are
relevant to our system, as well as a brief summary of the Spectre vulnerability.

\subsection{Out-of-Order Execution}
Many CPUs attempt to keep the pipeline full by executing instructions \emph{out of
order}, with the CPU allowing future instructions to be worked on and executed
while it waits for slower or stalled instructions to complete. To maintain
correctness and the original (Von Neumann) ordering, instructions are tracked in
a \emph{reorder buffer} (ROB), which keeps the order of instructions as they are
worked on out of order. Instructions are \emph{retired} from the ROB when
they are completed and there are no previous instructions that have yet to
retire. Upon retiring, an instruction's results are committed to the architectural
state of the CPU. Thus, the ROB ensures that the program (or debugger) view
of the CPU state always updates in program execution order, despite out of order
execution.

\subsection{Speculative Execution}

CPUs also attempt to keep their pipeline full by predicting the path of
execution. For example, a program may contain a branch that depends on a result
from a prior slow instruction. Rather than wait for the result, the CPU can
\emph{speculatively execute} instructions down one of the paths of a branch,
storing the results of the speculative instructions in the ROB.
If the guess of the branch target turns out to be correct, the CPU can quickly
retire all the instructions it has speculatively executed while waiting. If the
guess is incorrect, the CPU must discard the
(incorrectly) speculated instructions from the ROB, and continue executing from the
correct branch target.


\subsection{Branch Prediction}
When a CPU mispredicts a branch, the speculative execution results are
discarded, costing the CPU several cycles as the pipeline is flushed. To
minimize this, CPUs employ \emph{branch predictors} that attempt to guess the
path of execution. Branch predictors maintain a short history of previous
branch targets for a particular branch (e.g. whether a certain branch is
frequently taken or not taken), and use
this to inform the CPU's guess for speculative execution.

There are two kinds of branches a CPU handles: \emph{direct} and
\emph{indirect}. A \emph{direct} branch may either jump to a provided address or
continue executing straight through depending on the state of the CPU (e.g. condition
registers). While there are only two statically-known targets for a direct
branch, the CPU may not know if the branch is taken or not until preceding
instructions retire. An \emph{indirect} branch is always taken, but its address
is determined by the value of a register or memory address. Direct branches are
typically used for control flow such as \texttt{if} or
\texttt{for}/\texttt{while} statements, while indirect branches are used for
function pointers, class methods, or case statements.


%- BHB, BTB

\subsection{Spectre}

In early 2018, researchers revealed the Spectre vulnerability, which allows an
attacker to leak information from a victim
program~\cite{spectre,project_zero,maisuradze2018speculose}. Spectre uses the
fact that speculative execution can influence system state via side channel. In
Spectre, an attacker mistrains the branch predictor of a CPU running a victim
program by providing inputs to it. Once mistrained, the attacker then sends a
new input that will cause a different in-order execution path. However, because
the CPU's branch predictor has been mistrained, it will still speculatively
execute the previous path.

Consider the following code snippet from the Spectre
paper~\cite{spectre}:

\begin{lstlisting}
    if (x < array1_size)
        y = array2[array1[x] * 256];
\end{lstlisting}

The \texttt{if} statement correctly protects an out-of-bounds reads from \texttt{array1}.
But if the branch predictor makes an incorrect guess on the branch's direction
and speculatively executes inside the \texttt{if} statement, it may
cause a read beyond the boundary of \texttt{array1}. The result of this will
then (speculatively) be multiplied by 256 and used as an index into
\texttt{array2}. Although the CPU will not commit
the speculative update to \texttt{y}, it will still issue a memory read to
\texttt{array2[array1[x]*256]}, which will be cached. Importantly, even after
the CPU realizes the branch misprediction, it does not rollback the
state of the cache, as this does not directly influence program correctness.
However, the set of cached values is observable to the program via a
side-channel: by timing reads to \texttt{array2[i]}, the fastest read will
reveal the speculative value of \texttt{array1[x]*256}, for any value of
\texttt{x}. An attacker that is able to perform such a side-channel inference on
the cache can learn the speculative result of an out-of-bounds read from
\texttt{array1}.


Spectre can also be applied to indirect branches. Branch predictors use the
history of previous branches to predict the destination of an indirect jump when
the destination is not yet known. For direct branches, only one of two
destinations (taken or not taken) are possible to speculatively execute. But for
indirect branches, a mistrained branch predictor can potentially be coerced into
speculatively executing from \emph{any} target instruction in the binary.

We take advantage of the behavior of indirect branch prediction to hide the
location of our speculative computation.
%TODO more on why this is crucial...

%%%%%%%%%%%%%%
