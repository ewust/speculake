\pagebreak
\section{SPASM Instruction Set}
\label{appendix:spasm}
\renewcommand{\thefootnote}{\fnsymbol{footnote}}

SPASM is a 6 bit instruction set that we have developed to make use of the 
speculative primitive investigated in this work. The instruction set uses two 
functional registers, PTR and VAL, and a stack. This instruction set has been 
designed to be emulated in x86 and x86\_64 environments, though the emulator 
could be implemented to run on other systems. 

The progamming model used by SPASM builds up values on the maintained stack
and stores them in specially reserved offsets equivalent to x86 registers 
in order to perform system calls. Using the \texttt{UVAL} instruction
an immediate can be assigned into the lowest 4 bits of VAL, which 
can subsequenty be shifted to allow futher bits to be entered. In this way
immediates can be built up incrementally, stored, operated on, or pushed 
onto the maintained stack. Control Flow, including conditional jumps, 
is implemented by modifying the maintained instruction pointer SRIP. 

Using this we have implemented various example programs including 
\textit{HelloWorld}, \textit{FizzBuzz}, and \textit{ReverseShell}.
See Figure~\ref{fig:spasm_model} for a diagram of the SPASM emulator's
usage in an \speculake application.


\subsection{Instructions}
\begin{description}
  \setlength\itemsep{1em}
\item[UVAL] (0x3?) - 
The lower 4 bits of any instructions matching this wildcard are 
assigned onto the lowest 4 bits of VAL. This is used
for assigning immediates and computing offsets.
  
\item[SYSCALL] (0x1F) - 
Copies the values stored from the appropriate state struct registers 
into the equivalent physical registers and perform a syscall instruction. 
The result is copied back to the state struct in SRAX.

\item[BASE] (0x1E) - 
Set PTR to a \textit{uint64\_t*} pointing to the head of the state struct.  

\item[SHVAL] (0x1D) - 
Shift VAL to the left by 4 bits and store the result back into VAL. This is
used with UVAL to assign immediates.

\item[MREG] (0x1C) - 
Multiply VAL by 8 and store the result back into VAL. This is used to index 
into the state struct at the correct byte as the speculative register place
holders are at fixed offsets in the struct.

\item[PUSH] (0x1B) -
Push an item onto the stack maintained by the state struct and update the 
stack pointer SRSP accordingly. 
 
\item[POP] (0x1A) - 
Pop an item from the stack maintained by the state struct and update the 
stack pointer SRSP accordingly. 

\item[APTR] (0x19) - 
Set the value at the address in PTR equal to VAL.

\item[DPTR] (0x18) - 
Set VAL equal to the value at the address in PTR. 

\item[SWAP] (0x17) - 
Swap the values in the VAL and PTR.

\item[JZ] (0x13) - 
If VAL is equal to zero set SRIP equal to the value in PTR, else add one to SRIP.

\item[J] (0x12) - 
Assign SRIP to the value in PTR.

\item[CALL] (0x11) - 
Push the current SRIP onto the stack and set SRIP equal to the value in PTR.

\item[CMP] (0x10) - 
Compare the values in VAL and PTR, if PTR is greater than or equal to VAL
set VAL equal to 1; otherwise set VAL equal to 0. 

\item[ADD] (0x0F) - 
Add the values in PTR and VAL and store the result in PTR.

\item[2CMP] (0x0E) - 
Perform 2's compliment operation on VAL and store the result into VAL.
Used to implement subtraction operations.

\item[DADD] (0x0D) - 
Dereference the value in PTR, add VAL and store the result back into the dereference 
of PTR. 

\item[DMUL] (0x0C) - 
Dereference the value in PTR, multiply VAL and store the result back into the dereference
 of PTR. 

\item[DDIV] (0x0B) - 
Dereference the value in PTR, divide by VAL and store the result back into the 
dereference of PTR. 

\item[DSHL] (0x09) - 
Dereference the value in PTR, shift left by VAL and store the result back into the 
dereference of PTR. 

\item[DSHR] (0x08) - 
Dereference the value in PTR, shift right by VAL and store the result back into the 
dereference of PTR. 

\item[DAND] (0x07) - 
Dereference the value in PTR, bitwise and with VAL and store the result back into the 
dereference of PTR. 

\item[DOR] (0x06) - 
Dereference the value in PTR, bitwise or with VAL and store the result back into the 
dereference of PTR. 

\item[DXOR] (0x05) - 
Dereference the value in PTR, bitwise xor with VAL and store the result back into the 
dereference of PTR. 

\item[NOT] (0x04) - 
Perform a bitwise not on VAL and store the result back into VAL. 

\item[CLR] (0x02) - 
Set VAL equal to 0. 

\item[NOP] (0x00) - 
No-op instruction.
\end{description}

% \subsection{Programming Model}

% \subsubsection{Macros}
% \begin{description}
% \item[EUVAL] test
% \item[ECALL] test
% \item[EPUSH] test
% \item[EGET]  test
% \item[ESET]  test
% \end{description}

\renewcommand{\thefootnote}{\arabic{footnote}}
