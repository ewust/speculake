\documentclass[sigconf, anonymous]{acmart}



\usepackage{xspace}
\usepackage{listings}
\usepackage{siunitx}




\newcommand{\FigHighLevel}{

\begin{figure}[ht]
    \centering
    \includegraphics[clip, trim=6cm 13.5cm 3.8cm 5cm, width=\linewidth]{figures/exspectre-high-level.pdf}
    \caption{\textbf{Exspectre}\,---\, %
    1, 2, 3, 4...}

    \label{fig:high-level}

\end{figure}
}




\fancyhf{} % Remove fancy page headers 
\fancyhead[C]{Paper \#XXX} % TODO: replace 9999 with your paper number
\fancyfoot[C]{\thepage}

\setcopyright{none} % No copyright notice required for submissions
%\acmConference[Anonymous Submission to ACM CCS 2018]{ACM Conference on Computer and Communications Security}{Due 08 May 2018}{Toronto, Canada}
\acmYear{2017}

\settopmatter{printacmref=false, printccs=true, printfolios=true} % We want page numbers on submissions

%%\ccsPaper{9999} % TODO: replace with your paper number once obtained

\newcommand{\speculake}{Exspectre\xspace}


\begin{document}
\title{\speculake: Hiding Malware in Speculative Execution} % TODO: replace with your title

\begin{abstract}
Recently, the Spectre and Meltdown attacks have revealed serious vulnerabilities in modern CPU
designs, allowing an attacker to exfiltrate data from sensitive programs. These
vulnerabilities take advantage of speculative execution to coerce a processor to
perform computation that would otherwise not occur, leaking the resulting
information via side channels to an attacker.

In this paper, we extend these ideas in a different direction, and leverage speculative
execution in order to \emph{hide malware} from both static and dynamic analysis. Using this
technique, critical portions of a malicious program's computation can be shielded from view,
such that even a debugger following an instruction-level trace of the
program cannot tell how its results were computed.

We introduce \emph{\speculake}, which compiles arbitrary malicious code into a
seemingly-benign payload binary. When a separate trigger program performs a specific
pattern of indirect jumps, it (mis)trains the CPU's branch predictor. Subsequent
executions of the payload binary with similar jumps causes the CPU to mispredict its
branches and speculatively execute a malicious payload, which communicates
results back to the real world via side channels.

We study the extent and types of execution that can be performed in the
``speculative world'', and build tools that enable several computations to be
performed covertly. In particular, within speculative execution we are able to
decrypt memory using AES-NI instructions, and are able to interpret a custom
virtual machine language to perform arbitrary computation.
We also show how the trigger program
can be a pre-existing benign application already running on the system, and
demonstrate this concept with OpenSSL driven remotely by the attacker as our
trigger program.

Finally, we describe defenses, but in all honesty, you're probably not going to like
them.

We consider this to be a new thrust in attack vector research that harkens to work 
done on hardware red-pills and weird machines. We demonstrate that unintended 
functionality of architechture can be used to compose malicious programs with 
new and different properties. This technique poses novel and unique challenges for 
malware modeling efforts, as it forces an environment to either
faithfully reproduce all hardware behaviors, or overcome a much larger burden of 
reverse engineering. 


\end{abstract}

% TODO: replace this section with code generated by the tool at https://dl.acm.org/ccs.cfm

%\begin{CCSXML}
%<ccs2012>
%<concept>
%<concept_id>10002978.10003029.10011703</concept_id>
%<concept_desc>Security and privacy~Usability in security and privacy</concept_desc>
%<concept_significance>500</concept_significance>
%</concept>
%</ccs2012>
%\end{CCSXML}

%\ccsdesc{Security and privacy~Use https://dl.acm.org/ccs.cfm to generate actual concepts section for your paper}
% -- end of section to replace with generated code

\keywords{Malware; Covert Channel; Weird Machines; TK} % TODO: replace with your keywords

\maketitle


\section{Introduction}


Modern CPU designs use speculative execution to maintain high instruction
throughput, with the goal of improving performance. In speculative execution,
CPUs execute likely future instructions while they wait for other slower
instructions to complete. When the CPU's guess of future instructions is correct, the
benefit is faster execution performance. When its guess is wrong, the CPU
simply discards the speculated results and continues executing along the true path.


Previously, it was assumed that speculative execution results remain invisible
if discarded, as careful CPU design maintains strict separation between
speculative results and updates to architectual state. However, recent research
has revealed side channels that violate this separation, and researchers have
demonstrated ways to exfiltrate results from speculative computation. Most
notably, the Spectre vulnerability allows attackers to leak information from
purposefully mis-speculated branches in a victim process~\cite{spectre}. The
Meltdown vulnerability uses speculative results of an unauthorized memory read
to sidestep page faults and leak protected memory from the
kernel~\cite{meltdown}. Both of these vulnerabilities focus on extracting secret
data from a process or operating system. Recent follow-up work has revealed
other Spectre ``variants'',  including speculative buffer overflows, speculative
store bypass, and using alternative side channels besides the
cache~\cite{kiriansky2018speculative, spectre_ng}. In addition, several attacks
have leveraged Spectre to attack Intel's SGX~\cite{chen2018sgxpectre,
spectre_sgx, foreshadow}, and perform remote leakage
attacks~\cite{schwarz2018netspectre}.

\medskip

In this paper, we explore another attack enabled by speculative execution:
\speculake, which \emph{hides computation} within
the ``speculative world''. Taking advantage of the CPU's speculation to secretly
perform computation,
we can produce binaries that thwart existing reverse engineering
techniques. Because the speculative parts of a program never ``truly'' execute,
we can hide program functionality in the unreachable dead code in a program.
Even a full instruction trace, captured by a hardware debugger or software
emulator, will be unable to capture the logic performed speculatively.
This technique could lead to sophisticated malware that hides its behavior
from both static and dynamic analysis.


Existing malware use several techniques to evade detection and
make it difficult for analysts to determine payload behavior of reported malware. 
For example, binary \emph{packers} or \emph{crypters} encode an executable payload as
data that must be ``unpacked'' at runtime, making it difficult to tell
statically what a program will do~\cite{malware-packers}. Malware may also use
\emph{triggers} that only run the payload when certain conditions are present, preventing
it from executing when it is inside an analysis sandbox or 
debugger~\cite{balzarotti2010efficient,red-pill}.

However, with some effort, these existing malware techniques can be defeated.
Analysts can use
dynamic execution to unpack malware and reveal its
behavior~\cite{balzarotti2010efficient}, and can use symbolic execution or code
coverage fuzzers to determine the inputs or triggers that will reveal malicious
behavior~\cite{moser2007exploring,schwartz2010all,wang2017angr,egele2012survey}.


\speculake provides a new technique to malware authors, allowing them to hide
program functionality in code that appears to not execute at runtime. This technique
defeats existing static and dynamic analysis, making it especially difficult for
malware analysts to determine what a binary will do.


At a high level, \speculake consists of two parts: a payload program, and a
trigger program. When run by itself, the payload program executes a pattern of
indirect jumps and measures a cache-based side channel\footnote{We note that other
side channels could be used in place of a cache}. While the trigger program is
not running, the payload program is effectively inactive (potentially performing
a decoy operation). When the trigger program runs, it executes a similar
pattern of indirect jumps (with similar source and destination addresses as the
payload program), effectively training the CPU's branch predictor to the jump
pattern performed by the trigger program.

Importantly, the trigger program and payload program's indirect jump patterns
diverge on the destination of their final jumps. However, because the trigger
program has trained the CPU's branch predictor, the CPU speculates that the
payload program will continue following the pattern of the trigger program,
causing it to execute a \emph{speculative gadget} in the payload
program. This execution path is shown in Figure~\ref{fig:high-level}.

We emphasize that this gadget is in a region that is neither
read nor executed by the payload program, and after the CPU discovers the
mis-speculation in the payload program, it will discard the results from the
speculative gadget and continue executing from the correct destination. However,
the payload program has a limited speculative window where it can perform
computation, and can communicate results back to the ``real world'' payload 
program via a side channel.

%Figure~\ref{fig:high-level}


\FigHighLevel



While it's possible for the trigger and payload programs to be bundled in a
single program, an analyst aware of \speculake could easily find the speculative
gadget in the payload program based on the jump pattern in the trigger program.
However, if the trigger and payload are kept separate, the analyst has a much
harder job and must identify both.

Moreover, it is possible that the trigger program be a
seemingly-unrelated benign program already on the victim's computer. We
show this using the OpenSSL library as a benign trigger
program in Section~\ref{subsec:openssl}. If the trigger program is another benign
program on the system, the analyst has the difficult task of identifying which
program, library, or even operating system component is responsible for training
the CPU's branch predictor, and finding the specific set of jumps that occur at
runtime. To make matters worse, the payload program can include dummy jump
patterns that are effectively dead ends for the analyst, as they do not
ultimately call the hidden speculative gadget.

We also show it is possible to obviate the trigger program entirely, and
instead use \emph{trigger inputs}, which are attacker-provided data inputs to
the payload that cause the CPU to speculatively execute
at the attacker's chosen address. Unlike traditional malware input triggers, these inputs
cannot be inferred from the payload binary using static analysis or symbolic
execution, as the logic these triggers activate happens speculatively in the
CPU, which existing analysis tools do not model. We describe this technique in
more detail in Section~\ref{subsec:spec-buffer-overflow}.

Simulating or modelling the speculative execution path is a difficult task for a
program analyst hoping to reverse engineer an \speculake binary. First, the
analyst must reverse engineer and accurately model the closed-source proprietary
components of the target CPU, including the branch predictor, cache hierarchy,
out-of-order execution, and hyperthreading, as well as taking into account the
operating system's process scheduling algorithm. In contrast, the \speculake
author only has to use a partial model of these components and produce binaries
that take advantage of them, while the analyst's model must be complete to
capture all potential \speculake variants. Second, the analyst must run all
potential trigger programs through the simulator, including benign programs with
real-world inputs. Both of these contribute to a time-consuming and expensive
endeavor for would-be analysts, giving the attacker a significant advantage.

% Results summary
In order to study the potential of \speculake, we implement several example
payload programs and evaluate their performance. We find that the speculative
window is limited by the reorder buffer and physical register file sizes, which
allows us to execute between 100 and 200 instructions speculatively on modern
Intel CPUs. While brief, we show how to perform execution in short steps,
communicating intermediate results back to the ``real world'' part of the
payload program. Using this technique, we demonstrate implementing a universal
Turing machine (demonstrating arbitrary computation), a custom instruction set
architecture that fits within the constraints of speculative execution, and show
the ability to perform AES decryption of a text block via AES encrypt/decrypt
instructions.

Using these building blocks, we demonstrate the practicality of hiding arbitrary
computation by implementing a reverse shell in our speculative instruction set,
with instructions decrypted in the speculative world.
We show that this simple payload is able to perform several system calls in a
reasonable time, ultimately launching a dial-back TCP shell in just over
2~milliseconds after the trigger program starts.

%Using these building blocks, we demonstrate the practicality of hiding arbitrary
%computation by decrypting an AES-encrypted ARM binary in the speculative world
%one instruction at a time, returning the decrypted next instruction to the real
%world part of the payload program, which updates a virtual machine. We show that
%we are able to decrypt and process approximately 25 instructions per second.


% contributions?
% - explore the limits of speculative execution
% - propose novel concept of hiding computation in speculative execution
% - implement example applications using this concept, including decrypting
%   data speculatively
% - demonstrate using benign program (openssl) as trigger
% - identify defenses and discuss ways to counter them



% Move to discussion:
% TODO: we don't talk at all about how an analyst might try to enumerate
% the potential entry points to the speculative world, and how that can be
% made difficult




%%%%%%%%%%%%%%



\section{Background}
Modern CPU designs employ a wide range of tricks in order to maximize
performance. In this section, we provide preliminary background as they are
relevant to our system, as well as a brief summary of the Spectre vulnerability.

\subsection{Out-of-Order Execution}
Many CPUs attempt to keep the pipeline full by executing instructions \emph{out of
order}, with the CPU allowing future instructions to be worked on and executed
while it waits for slower or stalled instructions to complete. To maintain
correctness and the original (Von Neumann) ordering, instructions are tracked in
a \emph{reorder buffer} (ROB), which keeps the order of instructions as they are
worked on out of order. Instructions are \emph{retired} from the ROB when
they are completed and there are no previous instructions that have yet to
retire. Upon retiring, an instruction's results are committed to the architectural
state of the CPU. Thus, the ROB ensures that the program (or debugger) view
of the CPU state always updates in program execution order, despite out of order
execution.

\subsection{Speculative Execution}

CPUs also attempt to keep their pipeline full by predicting the path of
execution. For example, a program may contain a branch that depends on a result
from a prior slow instruction. Rather than wait for the result, the CPU can
\emph{speculatively execute} instructions down one of the paths of a branch,
storing the results of the speculative instructions in the ROB.
If the guess of the branch target turns out to be correct, the CPU can quickly
retire all the instructions it has speculatively executed while waiting. If the
guess is incorrect, the CPU must discard the
(incorrectly) speculated instructions from the ROB, and continue executing from the
correct branch target.


\subsection{Branch Prediction}
When a CPU mispredicts a branch, the speculative execution results are
discarded, costing the CPU several cycles as the pipeline is flushed. To
minimize this, CPUs employ \emph{branch predictors} that attempt to guess the
path of execution. Branch predictors maintain a short history of previous
branch targets for a particular branch (e.g. whether a certain branch is
frequently taken or not taken), and use
this to inform the CPU's guess for speculative execution.

There are two kinds of branches a CPU handles: \emph{direct} and
\emph{indirect}. A \emph{direct} branch may either jump to a provided address or
continue executing straight through depending on the state of the CPU (e.g. condition
registers). While there are only two statically-known targets for a direct
branch, the CPU may not know if the branch is taken or not until preceding
instructions retire. An \emph{indirect} branch is always taken, but its address
is determined by the value of a register or memory address. Direct branches are
typically used for control flow such as \texttt{if} or
\texttt{for}/\texttt{while} statements, while indirect branches are used for
function pointers, class methods, or case statements.


%- BHB, BTB

\subsection{Spectre}

In early 2018, researchers revealed the Spectre vulnerability, which allows an
attacker to leak information from a victim
program~\cite{spectre,project_zero,maisuradze2018speculose}. Spectre uses the
fact that speculative execution can influence system state via side channel. In
Spectre, an attacker mistrains the branch predictor of a CPU running a victim
program by providing inputs to it. Once mistrained, the attacker then sends a
new input that will cause a different in-order execution path. However, because
the CPU's branch predictor has been mistrained, it will still speculatively
execute the previous path.

Consider the following code snippet from the Spectre
paper~\cite{spectre}:

\begin{lstlisting}
    if (x < array1_size)
        y = array2[array1[x] * 256];
\end{lstlisting}

The \texttt{if} statement correctly protects an out-of-bounds reads from \texttt{array1}.
But if the branch predictor makes an incorrect guess on the branch's direction
and speculatively executes inside the \texttt{if} statement, it may
cause a read beyond the boundary of \texttt{array1}. The result of this will
then (speculatively) be multiplied by 256 and used as an index into
\texttt{array2}. Although the CPU will not commit
the speculative update to \texttt{y}, it will still issue a memory read to
\texttt{array2[array1[x]*256]}, which will be cached. Importantly, even after
the CPU realizes the branch misprediction, it does not rollback the
state of the cache, as this does not directly influence program correctness.
However, the set of cached values is observable to the program via a
side-channel: by timing reads to \texttt{array2[i]}, the fastest read will
reveal the speculative value of \texttt{array1[x]*256}, for any value of
\texttt{x}. An attacker that is able to perform such a side-channel inference on
the cache can learn the speculative result of an out-of-bounds read from
\texttt{array1}.


Spectre can also be applied to indirect branches. Branch predictors use the
history of previous branches to predict the destination of an indirect jump when
the destination is not yet known. For direct branches, only one of two
destinations (taken or not taken) are possible to speculatively execute. But for
indirect branches, a mistrained branch predictor can potentially be coerced into
speculatively executing from \emph{any} target instruction in the binary.

We take advantage of the behavior of indirect branch prediction to hide the
location of our speculative computation.
%TODO more on why this is crucial...

%%%%%%%%%%%%%%




\section{Architecture}


\speculake malware is comprised of two independent programs: a payload program,
and a trigger program. Both are installed on the victim's computer (e.g. via
trojan, remote exploit, or phishing), and must run simultaenously on the same
physical CPU. We note that the constraint of running programs on the same CPU is
not a signifcant burden: \texttt{taskset} can be used to limit a process to a
core, or if not available, the attacker can simply run multiple copies of the
trigger or payload program to coerce the OS scheduler to assign a
payload/trigger pair to the same core.

At a high level, the trigger program performs a series of indirect jumps in a
loop, training the branch predictor to this pattern. Meanwhile, the
payload program performs a subset of this jump pattern, then forces the CPU to
speculate by stalling the resolution of an indirect branch via a slow memory
read. The CPU will (mistakenly) predict the jump to follow the pattern performed
by the trigger program, and speculatively execute that destination in the
payload program.

\subsection{Indirect jumps}


In \speculake, we cause the CPU to mis-speculate the destination of an indirect
branch in the payload program, causing it to speculatively execute 
instructions that are never truly executed. We term this speculative destination
the \emph{speculative entry point}. \speculake uses indirect jumps to allow
speculative execution from \emph{any} instruction in the payload process'
address space. Because it can jump to any instruction, the malware analyst has a
difficult task in determining where a payload program's speculative entry point
is.

In fact, the location of this entry point is not determined by the payload
program, but rather the corresponding trigger program. This means that with only
the payload program, an analyst does not posses enough information to find the
speculative entry point.



Indirect branch predictors allow the CPU to predict the destination address of a
branch based solely off its source address and a brief history of previous
branch sources and destinations. While the inner-working details of modern CPU
branch predictors are proprietary, it is possible to reverse engineer parts of
their behavior, which we do for \speculake.

We observe that Intel CPUs consider three types of x86\_64 indirect branches:
\texttt{retq}, \texttt{callq *\%rax}, and \texttt{jmpq *\%rax}\footnote{other
general purpose registers besides \texttt{\%rax} can be used as well}.
We created a simple trigger program that performs a series of 28 indirect
branches
using \texttt{jmpq *\%rax} instructions. Between each jump, we increment
\texttt{\%rax} to account for the number of bytes between jumps. For the last
jump, we load a function pointer into \texttt{\%rax} and do a final indirect
branch using \texttt{callq *\%rax}.

In our payload program, we first perform the same 28 indirect jumps. We ensure
the source address of these jumps is the same as in the trigger program by
manually defining their containing function at a fixed address inside a linker
script. We also do the final indirect call to a function pointer, but with two
differences. First, the destination of the function pointer is a different
address, and second, the location of the function pointer in memory is uncached.
This forces the CPU to predict the destination of the indirect call while it
waits for the function pointer to load from memory. Due to the similar
history of branches with the trigger program,
the CPU will (incorrectly) predict the destination to be the same as the one in
the trigger program, which determines the speculative entry point for the
payload. Even though the in-order execution of
payload program never executes or even reads from this address, the CPU will
briefly execute instructions there speculatively.

Eventually, the dereference of the uncached function pointer in payload program
will be resolved, and the CPU will recognize it has incorrectly predicted the
destination of its \texttt{callq}/\texttt{jmpq}/\texttt{retq} instruction. The
results from the speculative entry point instructions will be discarded, and the
CPU will continue executing from the correct destination. However, as the
speculative code changes what is loaded into the cache based on its results, it
can covertly communicate its results to the ``real world'' program.



\subsection{Limits of Speculative Execution}

\FigCacheMiss

\FigSpecMeasure

We performed several experiments to determine how much computation can be
performed
speculatively, as well as what components are responsible for the limit.
We report results from our experiments on an Intel~Xeon-1270,
though we note we found similar results across other Intel processor generations,
including an i5-7200U, % jack
an i5-4300U, % ewust laptop
and an i5-4590. % ewust desktop
% For simplicity, we report results from only the Xeon-1270.



\subsubsection{Cache Miss Duration}
When executing instructions speculatively we rely on a memory load of a function
pointer from uncached memory. Thus, one potential limit on our computation comes
in the form of the time it takes for the memory read to return with a result
(and for the CPU to determine the result was mis-predicted).
We measured the number of cycles a
cache miss takes to return by artificially evicting an item from cache and
timing reads from its address.
Figure~\ref{fig:cache-miss} shows the CDF of cycles taken. In the typical case,
an evicted item takes approximately 300 clock cycles to load from the Level 3 cache (L3), which
would allow a limit of roughly 200 speculative instructions to be
executed during that time. We note that when an item is not in L3, it takes
considerably longer to load, in theory allowing for thousands of speculative
instructions in a significant fraction of runs.


\subsubsection{Reorder Buffer Capacity} \label{sssec:ROB}
We also measured the capacity of the reorder buffer (ROB) using a method
outlined by~\cite{measuring-rob}. We measure the maximum number of cycles taken
to perform two uncached memory reads, and vary the number of filler instructions
between them. If the number of filler instructions is small, both memory reads
will fit inside the ROB, and it can issue their memory reads in parallel.
However, if the filler instructions fill the ROB, the second memory read will
have to wait for the first to return before it can be issued, causing a
noticeable step increase in the cycle count. Figure~\ref{fig:spec-capacity}
shows this step occurs at approximately 220 instructions for our processor,
suggesting a hard upper bound regardless of how long the cache miss takes to
resolve.

%TODO: explain why it's noisy? why does the maximum number of cycles drop below
%this when we're above 220 instructions? Is the step somewhere above that, but
%other processes have crept in (but occasionally, we get lucky and it's just us)?

\subsubsection{Speculative Instruction Capacity}

To verify the upper limit of speculative instructions, we instrumented our
trigger and payload programs to test a simple gadget of variable-length before it
communicated a signal to the real world via a cache side channel. If the cache
side channel revealed no signal in the real world, then we know the speculative
execution did not make it to the signal instructions before the mis-speculated
branch was resolved.

We also tested whether instruction complexity or data dependencies impact the
number of instructions that can be completed. We find that data dependencies and
instruction complexity both have an impact on the number of instructions that
can be executed. Instruction complexity is determined by the number of $\mu$ ops
that the instruction uses, which appears to be what is tracked in the ROB. For
instance, on our architecture, the 64-bit \texttt{idiv} instruction takes 59
$\mu$ ops, and we can execute up to 3 of them in the speculative world.
Meanwhile, we can execute up to 18 32-bit \texttt{idiv} instructions, which each
take 9 $\mu$ ops~\cite{intel-instruction-tables}.

We found that the processor can identify some operations that have no effect on
output registers and allows them to use the entire size provided by the ROB, as
determined in Section~\ref{sssec:ROB}. This includes idomatic no-ops and zero
idioms, which have no reliance on register values, for example the instruction
\texttt{xchg \%rax, \%rax} and \texttt{xor \%rax, \%rax}. However, if an
instruction is a potential data hazard, it must use an entry in the PRF until it
reaches the correct stage in the execution pipeline. In this case, the number of
instructions could be limited by the PRF entries available.

%TODO: is this true? Or better explained by the micro-op story?

Most notably instructions that use the extended x86 registers are still valid
within the speculative context. Specifically, Intel's hardware accelerated
AES-NI encryption and decryption instructions, which each use 128-bit registers.
As shown in Figure~\ref{fig:spec-capacity}, speculative environments can
complete a significant number of AES rounds -- more than enough to decrypt a
full block using AES-CBC. We investigate the use of AES instructions in the
spculative environment further in Sections~\ref{subsec:decryption}.

\medskip

We find that when executing speculatively, the number of instrutions completed
is rarely limited by cache miss duration. Instead
ROB size and processor mechanisms for resolving
data dependencies define an upper boundary of approximately 150
instructions\footnote{While \texttt{nop} is able to execute up to the full 220
ROB capacity, instructions that do useful work cannot reach this limit}.

% TODO: 
% - Highlist instruction signal variablity (add v mul v aesdec v nop)?
% - Limitation is in the ROB (cite other work that supports this)
% - MORE PROCESSORS for testing?
% - ARM?


\subsubsection{Hyperthreading}

When running our tests, we assign the payload and trigger program to the same
core using \texttt{taskset}. We note in the absence of \texttt{taskset}, we can
run multiple instances of trigger programs to occupy all cores, eventually
having the payload program and trigger program become co-resident.

We also explore using hyperthreading, where the CPU presents two virtual cores
for each physical core, allowing the OS to schedule programs to each
simultaneously. In effect, this can cause the interleaving of instructions
between two programs to be much finer-grained: at the instruction level rather
than changing only at the OS-controlled context switch. We find that this
has two effects on speculative programs. First, the finer-grained interleaving
allows for a higher hit rate from the cache, suggesting that each indirect jump
pattern is more likely to result in speculatively executing from the intended
position. % TODO: can we quantify this?
Second, because the physical CPU is being shared, it effectively halves the
number of instructions that can be run in the speculative context.
Figure~\ref{fig:spec-capacity} shows the instructions that can be run when
running trigger and payload on a single core vs.\ a pair of hyperthreaded cores.


\subsection{Speculative Primitive}

We summarize our findings into a \emph{speculative primitive}, which allows us to
speculatively (and covertly) perform on the order of 100 arbitrary
instructions while an accompanying trigger program is running, and communicate
a short (e.g. single byte) result to the real
world via a cache side channel. These speculative instructions are able to read
from any real-world memory or registers, but they cannot perform updates or
writes directly. To update memory, the speculative instructions must communicate
to the real world through a cache side channel.

We note a performance tradeoff between the size of communication (e.g. 4 bits vs 8
bits) and the time it takes the real world to recover the result from the side
channel. Using Flush+Reload~\cite{yarom2014flush+} as our cache side channel,
recovering the result requires accessing all elements in an array exponential in
the size of the result (e.g. $2^8$ array reads to recover an 8-bit result). 
Therefore, there is a
performance advantage for keeping the size of the result small, and communicating
out small pieces of information that are aggregated by the real world.


%%%%%%%%%%%%%%



\section{Application Payloads}

While the amount of computation done in a single speculative execution is small,
we demonstrate several applications that can take advantage of multiple
speculative runs to carry out computation.

As a first step, we observe that the speculative primitive can be used to trivially
implement a finite state machine: any logic can be done in the speculative
world, while updates to the state are communicated to the real world where they
are stored. On the next run of the speculative instructions, they read from the
real world state (and other inputs), compute any state transitions, and
communicate the next state. In this mode, the state is maintained by the real world,
while updates are controlled by code executed speculatively.

Next, we observe that the instructions to be executed speculatively can be
obscured further by encrypting them. This encryption allows for hidden
computation, instructions are stored in the real world encrypted, they are then
decrypted in the speculative world and passed to the real world to be executed.
This method makes it appear as though the program is generating its own code.

\FigGeneralModel

\subsection{Turing Machine}
\label{subsec:turing}
To demonstrate that arbitrary computation can be done in the speculative world
we implement a Turing machine, configured to run the 5-state Busy Beaver 
function~\cite{chaitin1987computing,herken1992universal}.
The Busy Beaver is a configuration which writes the most
1s on the tape before halting with a limited number of states.

The computation of the Turing Machine is done speculatively, meaning that while
the real world keeps track of the state, tape, and symbols of the turing machine
all logic used to update the above is done in the speculative world.

\subsection{Unpacking and Decryption}
\label{subsec:decryption}

While a Turing machine demonstrates that arbitrary speculative computation
is possible, hiding malware this way has several drawbacks. First, an
analyst might be able to observe that something is driving a Turing
machine by noticing the real world state updating. With enough symbols, the
analyst could reverse engineer the state machine even without being able to see
the internal logic executing speculatively. Second, Turing machines are a
poor choice for practical computation, as they are inefficient and have no direct
way to perform system calls or otherwise interface externally.

We explore a more practical application of using \speculake to perform
decryption speculatively. To hide keys from the analyst, the key and decryption
code only occur in the speculative world, while committed state only learns the
resulting plaintext. Furthermore, the speculative code can
derive the decryption key based entirely off the trigger program, making it
impossible for an analyst that only has the payload program to
determine the key or decrypt the embedded ciphertext.


Any observer watching the execution of this program would never see any
indication of decryption, due to the exploitation of speculative execution. Any
observation of the executed instructions would not include any decryption
instructions, or loading of keys as they are only done speculatively, and thus
not shown in any debugging or real world traces of program execution. 
%Additionally the observer would be unable to find the executed instructions in
%the ELF file as they are only stored encrypted. 

We note that 200 instructions is too short for most software-implemented
cryptography. However, modern Intel CPUs provide hardware support for AES, which
we find only takes a handful of $\mu$-ops to perform the instructions needed in
AES decryption. We discuss details of our speculative AES decryption in
Section~\ref{subsec:impl-aes}.

\medskip


\subsubsection{Nested Speculation}

We explore the ability for the CPU to ``double speculate'', where a stalled
indirect jump while the CPU is speculating causes it to predict the target and
speculate a second time. For instance, suppose a payload program truly jumps to
target A, but the CPU is mistrained by a trigger program that jumps to B, thus
causing the payload program to speculatively execute at B. At B, suppose there
is a second indirect jump, perhaps using the same register as the first jump
(which has still not resolved). If the trigger program jumps to C, the payload
program will speculate a second time and also begin executing code at C.

We find that not all Intel CPUs support nested speculation. For
example, it appears Haswell chips do not speculate while already speculating,
but nonetheless support non-nested \speculake. Both Sandy Bridge (which
preceeded Haswell) and Kaby Lake (which followed Haswell) support nested
speculation.
% Could cut
We leverage this support in Section~\ref{subsec:decryption} to
increase the security of speculative decryption keys.

%There are numerous challenges for a reverse engineer to overcome to learn how a
%malicious program with this primitive works. As with previous examples, the
%malicious instructions are only executed when the correct trigger program is
%running (priming the jump table predictor \textbf{TK: word for this?}). However,
%when the to-be-executed instructions are executed the reverse engineer cannot
%locate these instructions in the ELF file, or generated source code. These
%instructions (when they even appear) seem to be generated by themselves, with no
%apparent cause. Additionally, there is no indicator that these instructions are
%encrypted as the decryption is done in the speculative world, meaning that no
%decryption (or key interaction) is done in a way visible to a reverse engineer.
%
%This method allows for arbitrary computation that is unobservable to any observer (static or dynamic). 


% 
% Argument for obfuscating keyschedule
% 
% Challenge for Rev-Engineer
%   - Find the obfuscated key schedule
%   - Find the correct trigger program

\subsection{Virtual Machines}
We demonstrate in Section~\ref{subsec:turing} that general purpose computation 
is possible in the speculative environment in spite of the constraints 
imposed. However, a Turing Machine
is not an efficient or usable model for computation when
attempting to accomplish tasks on modern processors. 

Using emulators in the ``real world'' allows for binary programs to be
placed encrypted and hidden in dead code and, as demonstrated in
Section~\ref{subsec:decryption}.
Traditional reverse engineering methods will reveal only that
emulation is being done, while the program being emulated remains encrypted.
Even when the trigger is running, only the parts of the code that execute would
be revealed to a careful analyst observing the CPU's committed state, and the
remainder of the emulated program would remain hidden.

We design a custom emulator and instruction set---SPASM (Speculative
Assembly)---that accommodates the constraints of the speculative primitive.
SPASM is a 6-bit Instruction set, where all instructions (including operand,
registers, and arguments) fit within 6-bits. This allows each step of the
speculative world to emit a single SPASM instruction to the real world for
emulation by a light-weight SPASM emulator. Using SPASM, developers can write
programs, assemble and encrypt them into a payload program. When the
associated trigger program runs, the payload will decrypt SPASM instructions in
the speculative world, and execute them one at a time.


% The assembly level abstraction that SPASM provides is still not extremely
% user friendly as it does not allow an author to write their encrypted 
% emulator payload in a high level language.  

While the custom emulator that we developed gives higher level abstraction to
an author, it still requires programs to be written in a custom assembly language. 
We note that the \speculake model is not intrinsically linked to the SPASM emulator. 
A wrapper could be implemented around other existing emulators to construct
instructions incrementally through the fixed-width channel (e.g. using 4 8-bit
reads to reveal a single 32-bit ARM instruction).
This would allow for encrypted payloads to be written in higher level languages
like C or C++, and compiled to any target architecture the real-world emulator
can execute.



\section{Implementation and Evaluation}
%We now discuss the details of our implementations and the information learned
%about speculative execution gained by these experiments. We discuss how we gain
%confidence in information passed back, and how a low number of iterations is
%sufficient to gain a strong confidence in results.

In this section, we discuss implementation details for our payload and trigger
programs.

%In each of our payload programs,
%we performed the same 28 indirect jumps and the final indirect jump with the
%function pointer with two differences: first, the function pointer was cleared
%from the cache (using the \texttt{clflush} instruction), and second, the
%function pointer pointed to a different destination (specifically, to a
%function
%that measured the cache side channel, recording the indexes in the probe array
%that appeared to be in the cache and then flushing all indexes).
%At address 0x430000, we placed the code for our speculative gadget within the 
%payload program, which is never directly called
%(but executes speculatively). We note that this code must be loaded into the
%cache but otherwise need not be accessed directly.
%


% Could move to an errata appendix?
%We randomized the stride length between indexes in
%this array to avoid the CPU's stride prediction from preemptively loading values
%in the buffer.

\subsection{Turing Machine}
\label{subsec:impl-turing}

We designed our Turing machine implementation to work with our custom trigger
program, with 28 indirect jumps mimicked by the Turing payload program.
We implemented a 2-symbol 5-state Busy Beaver Turing machine logic at the
speculative entry point (in 42 x86-64 instructions), returning the state update,
symbol to write, and tape move direction in a single byte via a cache side
channel.

We observed in our implementation that it is important that all values used in
the speculative world---as well as the code itself---be cached. If these are
evicted, the speculative code may fail to run, or the CPU may \emph{speculate}
on the value of the uncached item, which may be incorrect. While this does not
impact the correctness of normal programs whose incorrect speculations will be
resolved, our speculative code reports results back to the real world before
this resolution. In our Turing example, we observed this as incorrect state
transitions.

This error is particularly devious, as it is not an error of bit flips or noise,
but rather the processor speculating what the speculative gadget will read from
memory. Thus, error correcting codes on the reported result do not improve the
situation.


Instead, we repeat the execution several times and look for the modal value over
all iterations. We measured the error rate of our implementation as a function
of how many redundant iterations of the same step, and found that 10 redundant
iterations resulted in 1 error every million Turing steps, with the error rate
dropping exponentially as iterations increase. We choose 11 iterations as a
conservative bound (error rate measured to be 0), and computed 1 million Turing
steps at a rate of 1351 steps per second.

% Was 324 steps per second, but was that at 10 iterations? But we just said
% 10 wasn't perfect (1 error in a million steps), so it seems confusing to say
% we computed a million steps without error at 10 iterations.



% Target_fn returning next step given access to current state in global vars
%   - busy beaver
%   - cache page speculation on repetetive cycles. 
%       - mis-predict w/in speculative world 

% speed (complete smaller busy beaver?)
% against reference python implementation?

\subsection{AES Decryption}
\label{subsec:impl-aes}
The speculative world is able to take advantage of the AES-NI instructions to
decrypt messages. However, the speculative upper-limit of about 175 $\mu$~ops is
not enough to allow us to compute the key expansion, even using the
\texttt{aeskeygenassist} instructions. To avoid this, we can either preload the
expanded key schedule into the program (instead of the key), or use a cheaper
(non-standard) key expansion algorithm. For the former, we note that an analyst
could observe the structure of a normal key schedule, but we can avoid this by
simply selecting 11 random round keys. We note that this should not weaken the
security of AES, as we can ensure the round keys are not linearly related.


%We discuss further methods for obfuscating
%the key in Section~\ref{subsec:nested-spec}.
%

We wrote our AES decryption payload in 35 x86\_64 instructions and 2 lines of C
(which compiles to an additional 31 x86\_64 instructions). The payload
implements AES-CTR mode decryption, reading a global index and returning the
decrypted byte at that location in the ciphertext via the cache side channel.
In this model, the speculative function decrypts a full 16-byte AES block each
iteration, but only returns the bits specified by the index.


% TODO: this could move elsewhere?
We demonstrate the speed that information can be decrypted via the speculative
world, and we vary the channel width of the side channel from 1 to 12 bits to
measure its performance. At low channel width, reading from the cache side
channel requires timing reads from only 2 locations, while at 12-bits, the side
channel requires reading $2^{12}$ locations. On the other hand, there is a fixed
overhead per speculative iteration that favors increased channel width to
maximize bandwidth. As shown in Figure~\ref{fig:spec_bandwidth}, 8 bits is the
optimal side channel width, allowing us to decrypt over 5,000 bits per second
(625 Bytes/sec).
% Iterations: 20


\smallskip

We also implemented our nested speculation technique for obfuscating keys,
making 256 speculative landing spots that each shift 8 unique bits into the
128-bit register \texttt{\%xmm0}, and then performing an indirect jump. We then
had a custom trigger program perform 16 indirect jumps (after the initial 28)
that corresponded with 16 randomly-chosen landing spots in the payload program,
training the branch predictor. When the payload program reaches the first
speculative jump, it follows the same pattern speculatively, eventually filling
\texttt{\%xmm0} with the corresponding $16*8$ bits. We then used the
\texttt{aesenc} instruction to expand these 128-bits to a full key schedule, and
performed decryption as described previously. Thus, without the trigger program,
an analyst has no information about what key is used to decrypt the ciphertext
in the payload.



%While the \speculake model acheives only moderate decryption speed, the 
%critical computation of the decryption is done in the speculative world. 
%To further mask secure this decryption we make use of nested speculation 
%to obfuscate the key schedule, discussed in section~\ref{subsec:nested-spec}.

\FigSpecBandwidth

\subsection{Emulator}
\label{subsec:spasm}

% \subsubsection{SPASM}
% \label{subsubsec:spasm}

%Constructing an emulator making use of the speculative primitive requires  
%a trade off in expresive capability versus speed. 
We have implemented our custom instruction set architecture---SPASM---as a model
using two pseudo-registers, and 6-bit instruction length which allows for a
relatively direct programming model in which structured values can be entered
into memory locations before making a systemcall.


%To achieve a balance with speed the number of bits in each instruction is both 
%fixed and minimized.  A variable length instruction would require that the 
%\texttt{Flush+Reload} stage search the maximum number of bits on each round, and each 
%aditional bit doubles the search space that the \texttt{Flush+Reload} stage must 
%traverse.  So every bit shorter effectively doubles the throughput of the emulator~\footnote{Until it doesn't}, 
%and there is effectively no advantage to allowing variable length 
%instructions. 
%

In this model of computation there are effectively no instruction arguments, as
we must return an entire instruction from the speculative world inside the
limited-width cache side channel. Although other small instruction sets exist,
they either allow variable instruction lengths, are too long even in reduced
form, or did not have significant support to make them favorable for developers.

We used 6 bits in the construction of this instruction set as our goal is to
limit the length of each opcode as much as possible. Note that this is different
from the goal in maximizing bandwidth, as our goal now is to maximize
instruction throughput. Given our short instructions, loading values into
registers requires shifting in 4-bits at a time. SPASM has two registers that
act as a pointer and working register, that can be used to perform jumps,
arbitrary memory reads and writes, and basic arithmetic. We also have a
\texttt{syscall} instruction that makes a real system call to the underlying
operating system with parameters loaded from the SPASM state, allowing us to
interact with the real world. Details of the SPASM instruction set can be found
in Appendix~\ref{appendix:spasm}. 

In SPASM we have implemented multiple example programs that we encrypted and
loaded into a \speculake payload, which decrypts and emulates SPASM instructions
only when the corresponding trigger program is running. We have implemented a
\textit{HelloWorld} program that prints to \texttt{stdout}, and a
\textit{FizzBuzz} program that demonstrates control flow and arithmetic
operations while printing to stdout. Finally, we implemented a
\textit{ReverseShell} program that opens and connects a TCP socket to an
attacker-chosen location before executing a local shell and allowing the remote
adversary to issue shell commands on the victim machine.
Figure~\ref{fig:spasm_model} details the high-level flow of a SPASM payload.

%Expected performance for a given SPASM binary will vary given multiple factors,
%the most important of which is the \texttt{Flush+Reload} overhead. This can be
%tuned based on the specific trigger, as it is used to establish confidence that
%a signal has been identified in the indices returned from the "speculative
%world". 
%Thus accomplishing a task using SPASM generally equates to:

% \begin{lstlisting}
%     Redundancy * Probe_Space * Num_Instr
% \end{lstlisting}

Our \textit{ReverseShell} program consists of 355 SPASM instructions, and makes
six system calls to open a socket, connect to it, duplicate I/O file
descriptors, and perform an \texttt{execve} system call to open a shell. In our
tests using 5 iterations per decrypted instruction, the \textit{ReverseShell}
program takes just over 2ms to launch a reverse shell once triggered.

% TODO - REFERENCE APPENDIX for ISA

\FigSpasmModel

\subsection{OpenSSL Trigger}
\label{subsec:openssl-impl}

%Benign programs can also act as a trigger program, provided they perform sufficient
%indirect jumps to train the branch predictor. We experiment using the OpenSSL
%library as a potential benign trigger, as its source code has gratuitous use of
%function pointers which compile to indirect jumps. In addition, it has many
%complicated code paths that can be easily selected by remote clients by their
%choice of cipher suit

To demonstrate a benign trigger application, we implemented an ExSpectre payload
that would trigger when running concurrently with OpenSSL.
% Could cut?
We disable ASLR for simplification, but note that branch predictors can also be
used to determine ASLR offsets of co-resident applications, and our attack
adjusted accordingly~\cite{evtyushkin2016jump}.

We used \texttt{gdb} to run an instance of an OpenSSL server (version 1.0.1f),
and printed out every instruction executed and its address after a breakpoint on
the \texttt{SSL\_new} function. We then made a TLS connection to the server,
which produced over 13~million instructions, including over 359,000 direct jumps
and 28,000 indirect jumps. We then searched for the longest repeated set of more
than 28 indirect jumps that ends with a unique jump (i.e. source and destination
do not occur in the previous 28+ indirect jumps).

We discovered a candidate that corresponds to code in OpenSSL's
\texttt{nistp256.c} that contained 31 indirect jumps repeated 254 times each
handshake. This code is used during the TLS key exchange as the server computes
the ECDHE shared secret. We made a list of 31 source-destination address pairs
for these indirect jumps, and constructed a \texttt{jump}/\texttt{ret} chain to
mimic the same jump pattern in our payload program. Our payload program mimics
the first 30 indirect jump source/destination pairs, with a final jump going to
a cache timing function in our payload program. However, due to the prior
pattern, this last jump is frequently mis-speculated (about 3.5\% of the time),
and instead goes to the destination corresponding to the 31st jump in OpenSSL,
which serves as our speculative entry point.

%Our program jumps to the source address of a pair,
%which has a single \texttt{retq} instruction that pops the next
%destination off the stack, performing the indirect jump. At each destination
%address, we place another static (direct) \texttt{jmpq} to bring execution to the
%next source address in our list. The last jump source/destination pair is not
%performed (and its destination is not pushed to the stack). However, we place
%our gadget code at the destination address, which, when executed speculatively,
%loads a specific value into the cache.
%The ``true'' destination of the last indirect jump returns to our
%payload program, which immediately times probing the cache to see if any value was
%loaded.

We ran experiments on an Intel Haswell i5-4590 CPU, with OpenSSL and our payload
program pinned to the same core using \texttt{taskset}. We induced the jump
pattern in OpenSSL by running Apache benchmark against it to generate thousands
of TLS connections using the ECDHE key exchange with the secp256r1 curve
(ECDHE-RSA-AES256-GCM-SHA384). When running Apache benchmark locally, our
payload program reliably executes (speculatively) at the intended speculative
entry point about 3.5\% of the time. When apache benchmark runs on a remote
machine, this rate drops to approximately 2.0\%. Nonetheless, these are both
sufficient to perform computation, as our payload can simply increase the amount
of iterations needed to extract meaningful results from the speculative world.

We verified that our payload program did not execute at the speculative entry
point when we ran other programs that simply consumed CPU on the same core. In
addition, when we used Apache benchmark to create thousands of connections with
a different cipher suite (DHE-RSA-AES128-GCM-SHA256), we similarly saw no
speculation at the entry point. This could allow an adversary to use an obscure
or uncommon cipher suite to trigger a malicious \speculake payload program on a
remote server.


%\FigSpectreOne

\if0
\subsection{Speculative Buffer Overflow Trigger}
\label{subsec:sbo-impl}


To demonstrate the adaptability of this attack we have implemented the 
Speculative Buffer Overflow alternative trigger and assembled an \speculake 
payload. 

We make use of text input as the trigger to the payload program in this case,
where an input file contains entries specifying length, padding length,
and data value. When a line is read the payload program makes a bounds
check on the length of padding then stores the data at that offset in a
local array. 

The local array store is now the vulnerability it can be exploited even 
with complete bounds checks. After a series of honest inputs we include 
an entry with a padding length beyond the maximum. On the runs with this
cafted input we flush from cache the maximum padding length constant in 
the payload program, preventing the bounds check from being resolved 
and allowing the data value to be speculatively stored. We 
then speculatively overwrite the return address and use store-to-load forwarding
to cause a following return to divert to the address specified by the data entry. 
From here we make use of the decryption gadget to dcecrypt SPASM instructions
from an encrypted data blob to be transmitted to the real world via cache 
side-channel. Using this method we have been able to run all SPASM payloads 
including the reverse-shell. 

We note significant unpredictability in the branch predictor, though we
show in Figure~\ref{fig:spectre-one} that a minimum around 200 safe input entries
to each dishonest entry are required before a distinguishable signal is present 
in the cache channel. 
\fi
%%%%%%%%%%%%%%
 


\section{Discussion}


\subsection{Defenses}


\subsection{Future Work}
\label{subsec:future-work}





%%%%%%%%%%%%%%



\section{Related Work}
\label{sec:related-work}

% Many resourcess have been useful in developing the \speculake exploit 
% model.  

\subsection{Weird Machines \& Red-Pills}
Hardware side effects often leave behind unexpected state transitions 
that are not, or cannot be, accounted for in modeling environments. Malware
has historically made use of these bugs to create "weird" models of 
computation~\cite{weird_machines}. These state transitions can also leak 
information about the environment in which a program is running, which 
malware has used to detect honey-pots and emulated 
environments~\cite{balzarotti2010efficient}.

The term \textit{weird machine} comes from work done by Bratus \textit{et al.}
composing bugs and features of a system to create arbitrary compuatation with 
interesting properties~\cite{weird_machines,bratus2011exploit}. This 
encompasses many traditional computer vulnerabilities including 
buffer overflow~\cite{buffer_overflow}, format string exploits~\cite{format_string_exploit}, 
and return oriented programming~\cite{shacham2007geometry} to name only a few. 

Bangert~\textit{et al.} demonstrate that a turing complete weird machine 
can be constructed using the page fault handling mechanism in Intel's
IA32 architecture~\cite{bangert2013page}. Bugs like this are abundant in
modern processors~\cite{d2015exploiting} as multiple threads share system resources cooperatively
and optimizations is done objectively with regards of the 
process isolation for the most part~\footnote{With the exception of permission 
rings, i.e. kernel vs user space isolation}. Accessing cooperative hardware resources is at
the heart of the \speculake model, which constructs a hardware weird machine
using speculative branching and \texttt{Flush+Reload}. This is 
very much in the spirit of traditional exploit creation~\cite{weird_exploits}.

\smallskip 

Similarly hardware bugs have been used to identify and fingerprint execution
environments -- in malware this is know as a red-pill~\cite{red-pill} and 
is typically used to ensure that malware is not being emulated or 
debugged~\cite{lindorfer2011detecting, balzarotti2010efficient}. Paleari 
\textit{et al.} automate the process of discovering red-pills by applying 
fuzzing techniques to emulated environments and identifying situations 
in which the emulators differ from their physical counterparts~\cite{paleari2009fistful}.
This is still by no means an exhaustive list of red-pills for a given 
processor as many rely on timing or race conditions in physical 
environments. 

% Hardware bare-metal analysis
To combat this Kirat \textit{et al.} propose automated "bare-metal" 
emulation of malware~\cite{kirat2011barebox}, and methods to determine if malware
is performing red-pill checks to avoid emulation environments~\cite{kirat2014barecloud}.
Symbolic execution can also be used to combad red-pills by finding environmental triggers 
in malicious binaries~\cite{schwartz2010all}. 

\smallskip

Unlike other red-pills based on hardware bugs, \speculake cannot be bypassed
by patching binaries as the speculation can only happen on "bare-metal" systems, 
and the critical functionality is performed speculatively. Similarly, debugging 
break-points wll not be triggered as instructions are never completed in a committed state.
Symbolic execution will also fail to find a target function as it is never on the 
execution pathway. 



\subsection{Cache Side Channels}

The \speculake model relies critically on cache side channel attacks to 
communicate information from the speculative world. Cache side-channels have 
been used against a number of processors at essentially all different 
caching levels -- L1 data cache~\cite{percival2005cache,zhang2012cross,osvik2006cache}, 
L1 instruction cache~\cite{aciiccmez2010new}, 
L2 cache~\cite{ristenpart2009hey,percival2005cache}, 
LLC~\cite{ristenpart2009hey,liu2015last}, 
and branch prediction cache~\cite{aciiccmez2007power}. 
These attacks have also been performed in 
cloud~\cite{ristenpart2009hey,zhang2012cross}, 
browser~\cite{oren2015spy,google_cache_browser},
and security critical encryption~\cite{yarom2014recovering,tromer2010efficient}
environments, demonstrating how ubiquitous hardware side-channel exploits can be. 

Most cache side-channles fall into one of the following categories, although
some implement variations on these models.

\begin{itemize}
\item \texttt{Evict+Time} - An attacker with a synchronized clock evicts a series of items 
from cache. The victim then proceeds with computation, at which point the attacker can 
time accesses to each evicted item to identify which items a victim loaded~\cite{neve2006refined}. 
\item \texttt{Prime+Probe} - An attaker performs a computation, then evicts specific 
items from the cache before performing the same action again. Information about the cache
access patterns can be gained by comparing the timimg of the two 
computations~\cite{tromer2010efficient}.
\item \texttt{Flush+Reload} - Two cooperating processes  make use of the \texttt{Evict+Time} 
model to form a communication channel~\cite{yarom2014flush+}.
\item \texttt{Flush+Flush} - Follows the same model as \texttt{Prime+Probe},
however clflush is used instead of a load as it resolves much quicker in cases 
where the item is not in cache~\cite{gruss2016flush+flush}.
\end{itemize}

While other works have demonstrated higher throughput from similar 
covert cache side channels~\cite{liu2015last}, \speculake incurrs slightly higher performance 
penalties in our implementation as we perform redundancy checks to 
ensure the signal strength.  This redundancy combined with the intentional cache misseses
which enable speculative execution cost hundreds to thousands of clock cycles
per round. 


\subsection{Speculative Execution}

% Lead in 
Leading up to the discovery and release of the Spectre and Meltdown vulnerabilities
multiple teams were working to reverse engineer the mechanisms that implemented in 
the microcode of various processors and document their behavior~\cite{intel-instruction-tables, uop_article}.
Efforts to examine cache and memory security also preempted the spectre work, 
including Sophia D'Antoine's thesis work developing a side channel based on 
instruction interleavings in the CPU~\cite{d2015exploiting}. Similar research into branch 
predictors, cache replacement policies, and reorder buffer construction including work by Matt 
Godbolt~\cite{godbolt2016branch}, and Henry Wong~\cite{measuring-rob} paved the runway for 
dedicated branch prodiction side-channels.

% Discovery & Followup
Given the numerous directions that were being investigated in reverse engineering 
Intel processors, multiple teams discovered the Spectre vulnerabilites concurrently including 
the teams associated with Kocher \textit{et al.}~\cite{spectre}, the Project Zero team 
at Google~\cite{project_zero},and various others~\cite{evtyushkin2018branchscope,maisuradze2018speculose}. 
These works propose multiple variants making use of banch prediction.  Follow 
up work on speculative execution side channels have also demonstrated
web-based vulnerabilites~\cite{genkin2018drive} as well as hardware isolation  
leaks in Intel's SGX~\cite{spectre_sgx}. At the time of writing multiple sources
have claimed new speculative execution related exploits termed \textit{Spectre-NG}
have been reported to Intel and assigned CVE numbers, though we have no details
on content or severity.  


% defenses 
Much like cache side channel vulnerabilities speculative execution is relatively
critical to processor performance and requires significant software, and hardware 
updates to fully address. Multiple measures have been proposed to combat the 
individial variants of Spectre.   

The main defenses that have been widely applied to 
speculative branching attacks uses compiler level software patches to wrap sensitive 
function calls. Retpoline is Google's in house fix for enforcing 
isolation against branch target injection attacks~\cite{retpoline}. By crafting 
the stack around a sensitive function to trap into a pause instruction if the processor 
attempts to speculate past the wrapper data side-effects can be protected against. 
Similarly, Gruss \textit{et al.} propose the KAISER software patch to enforce address space
isolation for kernel and user space proceses~\cite{gruss2017kaslr}. 
Other proposals have considered hypervisor enforced time domain separation at varying 
granularity~\cite{renau2018securing}, and fuzzy clocks that would reduce the accuracy 
of any time based exfiltration pathway~\cite{hu1992reducing}. However, none of these defenses 
fundamentally defeat the use of a branch predictor as a cooperative mechanism
spanning multiple processes. 

Trippel \textit{et al.} proposes a method of faithfully simulating 
hardware bugs, such as processes cooperatively using speculating branch prediction,
by synthesizing microarchetecture defects~\cite{trippel2018meltdownprime}. However, 
manufacturers generally guard complete formal descriptions of processor microarchitecture closely, 
making this a less accessible solution for the research community.

\smallskip

We can certainly look forward to more vulnerabilities relating to speculative 
branching and related hardware side channels. Our work represents an exploit 
model for these vulnerabilities reliant upon shared hardware resources. 



%%%%%%%%%%%%%%


\section{Conclusion}


\bibliographystyle{ACM-Reference-Format}
\bibliography{paper}

\end{document}
